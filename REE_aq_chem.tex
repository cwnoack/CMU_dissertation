\chapter{Aqueous chemistry of the rare earth elements (REE)}\label{chap:REE_aq_chem}
\chaptermark{REE aqueous chemistry}

\section{What are the REE?}

The REE constitute much of Group 3 of the periodic table, a group of 16 transition metals, including the lanthanide series (La to Lu, excluding Pm), Yttrium (Y) and Scandium (Sc).
The ``rare'' moniker stems from their initial isolation from uncommon mineral phases in the 18th and 19th century \citep{CastorHedrick}, though the natural abundance of REE in the earth's crust range from 0.52 parts per million (ppm) to 41.5 ppm, in the same range as Pb or Sn and exceeding the natural, crustal abundance of Ag and Hg \citep{CRC}.

In the natural sciences, predictable thermodynamic differences between the REE make these elements uniquely capable tools for interpreting natural geologic and chemical processes \citep{Murray_Geol_1990, Laveuf_Geoderma_2009}.
Rare earth lithogeochemistries have long been used to infer depositional environments of geologic strata \citep{Murray_Geol_1990, PAAS, Hanson_AREPS_1980}.
Similarly, REE serve as benign analogs to the transuranic actinide series for nuclear waste disposal studies \citep{Krauskopf_CG_1986, Millero_GCA_1992};
as potential markers of regional authenticity for high value exported food products such as wine, pumpkin-seed oil, and olive oil \citep{Jakubowski_FJAC_1999, Joebstl_FC_2010, Farmaki_AL_2012};
and for studying mixing and metal cycling in the oceans \citep{DeBaar_Nature_1983, Elderfield_PTRS_1988}.

Many of the same properties that yield the unique and predictable geochemistry of the REE have lead to their use in more consumer products than nearly any other element group \citep{CastorHedrick, Graedel_PNAS_2015a}. In most applications, the performance of the REE is unmatched \citep{Ciacci_EST_2015, Nassar_JIE_2015}, making substitution (with more readily available/environmentally benign elements) undesirable.

Based on atomic number, the REE are segregated into light and heavy REE (LREE and HREE, respectively) with the division occurring between Eu and Gd \citep{CastorHedrick};
some studies further distinguish middle REE (MREE), though the specific elements are inconsistently defined between authors \citep{Hannigan_CG_2001, Tang_CG_2010, Choi_CG_2009}.
These ``weight'' distinctions allow for simplified description and quantification of the inter-element relationships, typically ratios of normalized concentrations, which are exploited in REE analysis.
Similarly, anomalies of certain REE --- due to redox lability for Ce and Eu \citep{Brookins_RMG_1989} and large anthropogenic emissions for Gd \citep{Bau_EPSL_1996} --- are used to interpret geochemical processes.
Y and Sc exhibit similar properties to the lanthanides and are thus included in the suite of REE with Y being most similar to HREE and Sc being most similar to LREE in solution \citep{Brookins_RMG_1989}.

\section{REE speciation in natural waters}

\bibliographystyle{unsrtnat}
\bibliography{Ch2_bib}