\chapter{Background}\label{CH2}

\section{The rare earth elements (REE)}

The REE constitute much of Group 3 of the periodic table, a group of 16 transition metals, including the lanthanide series (La to Lu, excluding Pm), Yttrium (Y) and Scandium (Sc).
The ``rare'' moniker stems from their initial isolation from uncommon mineral phases in the 18th and 19th century \citep{CastorHedrick}, though the natural abundance of REE in the earth?s crust range from 0.52 parts per million (ppm) to 41.5 ppm, in the same range as Pb or Sn and exceeding the natural, crustal abundance of Ag and Hg \citep{CRC}.

In the natural sciences, predictable thermodynamic differences between the REE make these elements uniquely capable tools for interpreting natural geologic and chemical processes \citep{Murray_Geol_1990, Laveuf_Geoderma_2009}.
Rare earth lithogeochemistries have long been used to infer depositional environments of geologic strata \citep{Murray_Geol_1990, PAAS, Hanson_AREPS_1980}.
Similarly, REE serve as benign analogs to the transuranic actinide series for nuclear waste disposal studies;13, 14
as potential markers of regional authenticity for high value exported food products such as wine, pumpkin-seed oil, and olive oil;15-17
and for studying mixing and metal cycling in the oceans.18, 19

Based on atomic number, the REE are segregated into light and heavy REE (LREE and HREE, respectively) with the division occurring between Eu and Gd;7
some studies further distinguish middle REE (MREE), though the specific elements are inconsistently defined between authors.20-22
These ``weight'' distinctions allow for simplified description and quantification of the inter-element relationships, typically ratios of normalized concentrations, which are exploited in REE analysis.
Similarly, anomalies of certain REE --- due to redox lability for Ce and Eu23 and large anthropogenic emissions for Gd24 --- are used to interpret geochemical processes.
Y and Sc exhibit similar properties to the lanthanides and are thus included in the suite of REE with Y being most similar to HREE and Sc being most similar to LREE in solution.23 

asdf

Aquatic geochemists apply the same principles used by geologists, the inter-element ratios and anomalies described previously, to infer water-rock interactions, hydrologic communication between geologic units, and groundwater mixing members.25-28
Interactions with different mineral phases have been shown to alter REE patterns predictably.
For example, a MREE enrichment is observed for fresh waters in contact with phosphate-rich minerals20 while HREE enrichment is found in carbonate-rich waters.29 
Tesmer, et al.30 showed that groundwater communication, especially at shallow depths, could be established by interpretation of REE patterns.
Similarly, multiple authors have used REE concentrations to calculate end member contributions to a groundwater flow.25, 27
The capabilities of REE to serve as tracers have wide applications to any system concerned with fluid migration, fluid mixing, and water-rock interactions.

\subsection{Commercial usage of the REE}

\subsection{Traditional routes of REE extraction and recovery}

\subsection{Aqueous chemistry of the REE}


\section{Resource recovery from aqueous media}

\subsection{Geothermal waters}

\subsection{Oil \& gas produced waters}


\section{Aqueous resource recovery strategies}
\sectionmark{Recovery strategies}


\section{Economic viability of low-concentration mineral resources}
\sectionmark{Resource economics}

\bibliographystyle{unsrtnat}
\bibliography{Ch2_bib}