% Expected contributions and broader impacts
\chapter{Major contributions and broader impacts}
\vspace{-1cm}

At the conclusion of this research, we expect to have contributed significant knowledge to the field of REE geochemistry, in particular, as it applies to saline fluids.
While the context of this work primarily relates to geothermal waters, the potential for extraction of REE from aqueous media is far reaching.
Other large-scale engineering processes such as carbon capture, utilization, and storage (CCUS), desalination waste-brine management, or oil and gas production handle large volumes of highly saline waters that may contain valuable quantities of REE.
Moreover, highly selective adsorbents would represent a significant improvement compared to the relatively non-specific ligands utilized in solvent extractions during traditional REE production schemes.

More specifically, each objective of this research is expected to improve understanding of REE geochemistry.
In Objective 1, the compilation and analysis of literature data highlighted the enormous variability in dissolved REE concentrations, while identifying the lack of knowledge regarding the controls on REE abundance and the occurrence of REE in hypersaline solutions.
Objective 2 will provide a robust and rigorously validated technique for filling in this gap, allowing for reliable determination of the REE in brines, without requiring large sample volumes.
Finally, Objective 3 provides systematic study of the effects of surface attachment of ligands, with known aqueous REE affinity, for use in an extraction and recovery scheme.
As yet, economically viable alternatives to traditional REE extraction are almost non-existent.
This work represents the early stages of developing a technology with the potential to generate a consistent domestic supply of materials critical to green energy and defense technologies; valorize high-volume wastes or low-value industrial byproducts; and avoid environmental impacts from primary REE mining..
