% Introduction
\chapter{Research plan}
\vspace{-1cm}

Four specific objectives were developed for this work, combining literature review, data analysis, analytical technique development, focused experimentation, and modeling:

\begin{itemize}
	\item Determine REE abundance and trends in natural waters.
	\item	Develop efficient liquid-liquid extraction technique for separation of REE from hypersaline brines
	\item Study the effects of ligand functionality and geometry on the partitioning behavior of the REE from brines to functionalized adsorbents
\end{itemize}

These individual tasks are described subsequently in \S 2.1--2.4.

\section{Determine REE abundance and trends in natural waters}

In this objective, a compilation and analysis of data from numerous independent studies of REE in natural waters was performed.
The compiled data were used to develop a consistent database of REE concentrations and their associated major solute chemistry and to explore interelement relationships, examine trends in REE abundance, and test hypotheses related to REE abundance as functions of major solution chemistry parameters.
The tasks within this objective were: (1) to ascertain an expected range of dissolved REE concentrations in waters of variable chemistries, (2) to derive unbiased estimates of REE distributions, and (3) to investigate trends in REE abundance in groundwater in relation to other available chemical parameters (e.g. pH, ionic strength, and major solution species).
Results of this work have been published and described in detail in Noack, et al.50 which is included in Appendix A.1.

The major findings of this work were that the REE are found in natural waters across ten orders of magnitude of concentrations, that pH appears to be the only variable at the ``macro'' scale of this study which significantly influences abundance, and that, while limited data exist for REE in brines, the REE composition of brines is potentially unique.
These results have implications on the remainder of this project.
First, the natural variability and lack of data of REE in brines necessitates development of robust, reliable analytical techniques for such waters as well as the application of this technique to hypersaline brine samples.
Second, the lack of consistent predictors of REE abundance requires focused experimentation of REE source and sink behavior in the environments of interest.

\section{Develop a liquid-liquid extraction technique for separation and concentration of REE from brines}

In this objective, REE separation and preconcentration from highly saline brines using a liquid-liquid method was studied.
A significant limitation of published methodologies for REE quantification in aqueous samples is the lack of validation of the methods in systems with high salinity and dissolved metals.
A common ligand used for REE complexation and extraction, bis(2-ethylhexyl) phosphate (HDEHP), was studied in a heptane diluent. 
The tasks of this objective were to: (1) demonstrate feasibility of REE recovery from small volumes of hypersaline brines by LLE, (2) optimize LLE methodology for high salinity and metals content, and (3) validate method for synthetic brines of varying complexity.
Results of this work have been published and described in detail in Noack, et al.50 which is included in Appendix A.2.

The major finding of this work was that the REE are measurable at environmentally relevant concentrations in hypersaline solutions with high accuracy using small volumes of samples.
Moreover, the method is robust to variability in salt, dissolved organic carbon, and competing metal concentrations.
This method can be confidently applied to the analysis of natural samples in the future for calibration of engineered recovery systems.

\section{Study REE partitioning to novel functionalized adsorbents from saline solutions}