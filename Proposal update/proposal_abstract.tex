\chapter{Abstract}
The rare earth elements (REE) are a suite of 16 elements with coherent and predictable chemistries with ubiquitous utilization in modern technology.
While not truly rare in average abundance, the lack of concentrated deposits and the challeneges of intra-group separations make these elements highly valuable and difficult to produce in mass.
Large economic and environmental costs associated with primary REE extraction motive the development of alternative REE resources, such as concentrated brines.
This thesis will focus on the occurrence of REE in hypersaline fluids and the separation of the REE from these fluids.

The REE hare currently produced exclusively through primary mining activities.
However, these methods pose significant environmental risk and occur almost exclusively in China.
A technology capable of exploiting alternative REE resources domestically would be environmentally, economically, and geopolitically important.

Geothermal fluids represent a promising target for validating REE extraction and recovery from complex aqueous systems.
A typical geothermal power plant handles tens to hundreds of thousands of gallons per day per MWe of brine with known, elevated concentrations of the REE.
However, significant challenges exist for the extraction of the REE given their dilute concentrations (as compared to the major constitutents of the brine) and, in turn, the large excesses of non-REE cations capable of competing for the reactive sites of an extraction technology.

The goals of this research will be to evaluate the feasibility of extracting and recovering the REE from saline fluids.
The development of this technology will be supported through focused experimentation and development of analytical techniques along with consideration of previously published data and adaptation of existing recovery schemes.

The goals of the research will be achieved through four specific objectives.
The first objective of this work will be to accumulate available water quality data with REE measurements from the literature and explore relationships among the REE as well as between the REE and other commonly measured, bulk analytes.
The second objective will be to develop and validate an efficient liquid-liquid extraction for determination of REE in hypersaline brines.
The third objective will be to study the effects of ligand functionality and geometry on the partitioning behavior of the REE from brines to functionalized adsorbents.
The final objective will be to study the best performing adsorbent(s) in detail.
Taken together, these objectives will be a contribution to the understanding of trace-metal geochemistry, particularly in high salinity systems, and an assessment of extraction and recovery strategies targeted at the REE in hypersaline fluids.
