\chapter{Abstract}
The rare earth elements (REE) are a suite of 16 elements with coherent and predictable chemistries with ubiquitous utilization in modern technology.
While not truly rare in average abundance, the lack of concentrated deposits and the challeneges of intra-group separations make these elements highly valuable and difficult to produce in mass.
Large economic and environmental costs associated with primary REE extraction motive the development of alternative REE resources, such as concentrated brines.
This thesis will focus on the occurrence of REE in hypersaline fluids and the separation of the REE from these fluids.

The development of the Marcellus shale for natural gas provides an excellent case study for the use of REE in source identification and apportionment.
Both the character (i.e. elevated salinity, dissolved heavy metals, naturally occurring radioactive material) and volume (millions of liters per drilled well) of waste brines associated with the oil and gas extractive industry provide sufficient motivation for investigation of advanced monitoring tools for protection of freshwater resources.
Similarly, the wide variety of contaminant sources (i.e. having the potential to degrade high quality waters) in Southwestern Pennsylvania --- including conventional oil and gas wastewater, abandoned mine drainage, varied industrial wastes, municipal waste, and agricultural run-off --- further promote the search for analytes which would allow for discrimination among potential sources.

The REE have previously been used to hypothesize groundwater mixing and sources of salinity in steady-state systems.
However, these methods have either lacked mathematical rigor (i.e. are qualitative and typically based on visual inspection of REE profiles) or have failed to consider variability in source signatures while assuming conservative behavior (i.e. deterministic).

The goals of this research will be to evaluate the feasibility of using REE as tracers of saline groundwaters, and to determine the appropriate mathematical and statistical models for interpreting REE data for environmental forensics.
The development of these models will be supported through focused experimentation and development of analytical techniques along with consideration of previously published data and adaptation of existing source apportionment models.

The goals of the research will be achieved through four specific objectives.
The first objective of this work will be to accumulate available water quality data with REE measurements from the literature and explore relationships among the REE as well as between the REE and other commonly measured, bulk analytes.
The second objective will be to develop and validate an efficient liquid-liquid extraction for determination of REE in hypersaline brines.
The third objective will be to study the effects of ligand functionality and geometry on the partitioning behavior of the REE from brines to functionalized adsorbents.
The final objective will be to study the best performing adsorbent(s) in detail.
Taken together, these objectives will be a contribution to the understanding of trace-metal geochemistry, particularly in high salinity systems, and an assessment of extraction and recovery strategies targeted at the REE in hypersaline fluids.
