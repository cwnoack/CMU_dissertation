\chapter{Introduction, problem identification, and research goals}
\chaptermark{Introduction}

%\clearpage

\section{Introduction}

Modern technologies --- including catalysts, high-strength alloys, high-efficiency phosphors, lasers, and magnets --- are dependent upon the unique properties of the rare earth elements (REE) for their efficacy of operation.
However, global REE material-flows are prone to complex environmental, technical, and geopolitical forces on both the supply- and demand side.
Development of economically-viable technologies for the extraction of the REE and other critical materials from unconventional sources (such as geothermal fluids, oil and gas produced waters, or coal combustion residuals) has great potential value to:
generate a consistent domestic supply of materials critical to green energy and defense technologies;
valorize high-volume wastes or low-value industrial byproducts;
and avoid environmental impacts from primary REE mining.

This research seeks to address the potential for REE extraction and recovery from dilute aqueous sources.
A combination of a literature analysis, analytical method development, controlled experimentation, and statistical and geochemical modeling will be used to address these goals.

This thesis is arranged into distinct chapters to address key knowledge- and technical gaps in the existing literature.
Chapter 1, this introductory chapter, provides brief context to the state of the REE-economy, elucidating the benefits of developing alternative resources.
\hyperref[CH2]{Chapter~\ref*{CH2}} provides a background on the aqueous geochemistry of the REE, with a focus on complex, high-salinity systems.
\hyperref[CH3]{Chapter~\ref*{CH3}} describes the results of a review and compilation of published REE data, accompanied by an in depth statistical analysis of occurrence distributions and trends.
\hyperref[CH4]{Chapter~\ref*{CH4}} details the optimization and application of a liquid-liquid extraction to precisely measure REE concentrations in small volumes of complex, hypersaline fluids.
\hyperref[CH5]{Chapter~\ref*{CH5}} characterizes the performance of surface attached ligands for the extraction and recovery of the REE from synthetic brine solutions.
Finally, \hyperref[CH6]{Chapter~\ref*{CH6}} summarizes this project in the context of alternative REE resources and suggests future work.

\section{Problem identification}

The constantly increasing consumer products incorporating the REE, and the ensuing demand for these products, have established the REE as valuable global commodities.
Domestic demand in 2012 was 11,300 tons, while the global demand was more than 113,000 tons \citep{FrostSullivan_REEmarket}.
Much of that demand is a result of a booming green energies market.
In particular, the permanent magnets sector is expected to experience significant growth between 2013 and 2020.
High-efficiency generators used in turbines and electric motors require strong and light permanent magnets.
Currently, magnets using neodymium, praseodymium, and samarium (with dysprosium and terbium additives) are the strongest and lightest commercially available.

Even at the height of domestic REE production at Mountain Pass in CA (2012), the U.S. imported \$520MM worth of REE compounds, with nearly 40\% coming from China, in order to support a demand of 11,300 tons \citep{USGS_minyb_2012}.
A booming green energies industry fueled, and continues to fuel, both domestic and global demand, where the REE provide superior performance and efficiency compared to alternative materials \citep{Nassar_JIE_2015, Graedel_PNAS_2015}.
Sustained growth in these technologies is dependent on diversifying REE sources given projected supply shortages in the medium- to long-term.
Traditional mining of REE from ore deposits is unlikely to fulfill this demand for technical and economic reasons, especially in the US where traditional mining of REE is suspended.
China, the world's leading source of REE and primary supplier of global demand for many years (Figure~\ref{fig:world-REO-prod}, has indicated partial control of exports, or punitive tariffs, in the near future.
This emphasizes the need for REE separation and recovery from unconventional matrices, such as geothermal fluids.

\begin{figure}[htbp]
\begin{center}
\includegraphics[width = \textwidth]{Ch1_figures/World-REO-production.pdf}
\caption{Global, primary REE production, 2000-2014.
Data from \citet{USGS_commsumm}.}\label{fig:world-REO-prod}
\end{center}
\end{figure}

Global REE reserves are estimated at 130 million metric tons \citep{USGS_commsumm}, much of which is located in low-concentration deposits or ocean-floor manganese nodules, which are both extremely expensive to mine with current methods.
This limits the number of readily mineable REE deposits and, ultimately, our ability to increase REE supply \citep{JRC_2011, Alonso_EST_2012}.
Aqueous media such as brines or produced waters from geothermal energy, conventional oil/gas, and shale gas extraction operations are potentially significant, but unexplored sources of REE.

Presently, REE extraction is accomplished by traditional mining (e.g. open-pit) followed by chemically-, and energy-intensive element separations, which incur a significant environmental burden (Figure~\ref{fig:Zaimes-LCA}; \citep{Zaimes_SCE_2015}).
Even when present in ores at appreciable levels, REE are commonly commingled with radioactive thorium and uranium, which need to be safely separated and stored in addition to standard waste management associated with mining (e.g. tailings) \citep{Gupta_IMR_1992, Sprecher_EST_2014}.
Stringent environmental regulations, time-intensive processes, and expensive permits complicate the opening of new, domestic mines because of these inherent risks.
On this basis, projections expect that exploiting traditional REE sources to meet increasing demand will be a significant challenge.
In 2012, China was responsible for more than 95\% of the global REE supply \citep{USGS_commsumm}.
China also had the largest demand for REE, at 66\% of the total global demand \citep{USGS_commsumm}.
The US was the next largest consumer, at 10--15\% of the total demand.
In 2011, China announced a 35\% reduction in exports of REE, in an effort to meet their domestic needs \citep{USGS_minyb_2012}.
This created large instability in the REE market as there were no other major sources for REE \citep{Alonso_EST_2012, Chakmour_Elem_2012, Hatch_Elem_2012}.
China is expected to continue reduction of exports, through either quotas or tariffs, as a mean to reduce stress on its REE reserves \citep{FrostSullivan_REEmarket, USGS_commsumm}.

\begin{figure}[htbp]
\begin{center}
\includegraphics[width = \textwidth]{Ch1_figures/Zaimes-LCA-schematic.pdf}
\caption{Traditional REE production flowsheet (left) and the associated fraction of lifecycle environmental impacts for the boxed boundary (right) using data from \citet{Zaimes_SCE_2015}.
Only selected impact categories are shown, however the dashed line (72\%) indicates the overall average across ten categories.}\label{fig:Zaimes-LCA}
\end{center}
\end{figure}

Primary mining effectively constitutes 100\% of the REE production worldwide \citep{Binnemans_JCP_2013, USGS_commsumm},
but interest in recovery of REE from end-of-life stocks (EOL), from unconventional resources, and from REE-containing industrial wastes has expanded rapidly in recent years \citep{Binnemans_JCP_2015}.
High volumes of REE are deployed in permanent magnets, while high-value REE are used in phosphors \citep{Hatch_Elem_2012},
making these products two primary targets for recycling from EOL products along with metal hydride batteries \citep{Binnemans_JCP_2013,Tunsu_Hydro_2015}.
REE are applied in many other products, but the REE content is dissipated in-use or rendered unrecyclable by current designs \citep{Ciacci_EST_2015}.
Ferrous shredder waste (where magnets could accumulate) is a promising potential resource for REE and other critical materials.
However, \citet{Bandara_JSM_2015} propose that recycling of ferrous shredder-waste would need to exceed 50\% in order to dampen Nd price volatility from recycling alone. The conclusion from these forecasts is the need for novel, alternative feedstocks \citep{Bandara_JSM_2015}.

\section{Research goals}

The three, specific objectives of this work and the related questions they were developed to answer are as follows:
\begin{enumerate}
\item Determine REE abundance and trends
	\begin{itemize}
		\item What is the natural variability of REE in aqueous media?
		\item What quantitative methods exist for considering below-detection-limit values?
		\item What is known about REE occurrence in brines?
		\item What relationships between REE and bulk solution properties are important for REE fate and transport?
	\end{itemize}
\item Develop efficient liquid-liquid extraction technique for separation of REE from hypersaline brines
	\begin{itemize}
		\item How are REE measured in aqueous samples
		\item Will these techniques work for hypersaline, chemically complex brines?
		\item What are the limits of effectiveness for a LLE technique (with respect to brine composition)?
		\item What relationships between REE and bulk solution properties are important for REE fate and transport?
	\end{itemize}
\item Study the effects of ligand chemistry and geometry on REE partitioning from saline solutions to functionalized adsorbents
	\begin{itemize}
		\item How are REE measured in aqueous samples
		\item Will these techniques work for hypersaline, chemically complex brines?
		\item What are the limits of effectiveness for a LLE technique (with respect to brine composition)?
		\item What relationships between REE and bulk solution properties are important for REE fate and transport?
	\end{itemize}
\end{enumerate}


\bibliographystyle{unsrtnat}
\bibliography{Ch1_bib}