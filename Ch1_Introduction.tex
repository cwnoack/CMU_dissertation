\chapter{Introduction, problem identification, and research goals}
\chaptermark{Introduction}

\clearpage

\section{Introduction}

Rare earth elements (REE) constitute a group of chemically similar metals in the lanthanide series, as well as yttrium and scandium, which are essential components of modern technologies from magnets to batteries.
Their extensive use as chemical catalysts, in metallurgy and alloys, glass polishing and other sectors of manufacturing make them an absolute necessity for advanced technologies and materials \citep{USGS_commsumm}.

\section{Problem identification}

The constantly increasing consumer products incorporating the REE, and the ensuing demand for these products, have established the REE as valuable global commodities.
Domestic demand in 2012 was 11,300 tons, while the global demand was more than 113,000 tons \citep{FrostSullivan_REEmarket}.
Much of that demand is a result of a booming green energies market.
In particular, the permanent magnets sector is expected to experience significant growth between 2013 and 2020.
High-efficiency generators used in turbines and electric motors require strong and light permanent magnets.
Currently, magnets using neodymium, praseodymium, and samarium (with dysprosium and terbium additives) are the strongest and lightest commercially available.

Although the REE supply is expected to remain even with demand for the next 5-7 years, a supply shortage is expected in the longer term, as demand for technologies using REE increases [REF].
Therefore, sustained growth in these technologies is dependent upon the diversification of REE sources to meet demand.
Traditional mining practices used for extraction and enrichment of REE from ores are unlikely to fulfill this demand for technical and economic reasons.
This emphasizes the need for REE separation and recovery from unconventional matrices.

Global REE reserves are estimated at 130 million metric tons \citep{USGS_commsumm}, much of which is located in low-concentration deposits or ocean-floor manganese nodules, which are both extremely expensive to mine with current methods.
This limits the number of readily mineable REE deposits and, ultimately, our ability to increase REE supply \citep{JRC_2011, Alonso_EST_2012}.
Aqueous media such as brines or produced waters from geothermal energy, conventional oil/gas and shale gas extraction operations are potentially significant, but unexplored sources of REE.

Presently, REE extraction is accomplished by traditional mining (e.g. open-pit) followed by chemically-, and energy-intensive element separations, which incur a significant environmental burden [REF].
Even when present in ores at appreciable levels, REE are commonly comingled with radioactive thorium and uranium, which need to be safely separated and stored in addition to standard waste management associated with mining (e.g. tailings) [REF].
Stringent environmental regulations, time-intensive processes, and expensive permits complicate the opening of new, domestic mines because of these inherent risks.
On this basis, projections expect that exploiting traditional REE sources to meet increasing demand will be a significant challenge.
In 2012, China was responsible for more than 95\% of the global REE supply [REF].
China also had the largest demand for REE, at 66\% of the total global demand [REF].
The US was the next largest consumer, at 15\% of the total demand.
In 2011, China announced a 35\% reduction in exports of REE, in an effort to meet their domestic needs [REF].
This created large instability in the REE market as there were no other major sources for REE.
China is expected to continue reduction of exports, through either quotas or tariffs, as a mean to reduce stress on its REE reserves \citep{FrostSullivan_REEmarket, USGS_commsumm}.

\section{Research goals}

\begin{enumerate}
\item Determine REE abundance and trends
	\begin{itemize}
		\item What is the natural variability of REE in aqueous media?
		\item What quantitative methods exist for considering below-detection-limit values?
		\item What is known about REE occurrence in brines?
		\item What relationships between REE and bulk solution properties are important for REE fate and transport?
	\end{itemize}
\item Develop efficient liquid-liquid extraction technique for separation of REE from hypersaline brines
	\begin{itemize}
		\item How are REE measured in aqueous samples
		\item Will these techniques work for hypersaline, chemically complex brines?
		\item What are the limits of effectiveness for a LLE technique (with respect to brine composition)?
		\item What relationships between REE and bulk solution properties are important for REE fate and transport?
	\end{itemize}
\item Study the effects of ligand chemistry and geometry on REE partitioning from saline solutions to functionalized adsorbents
	\begin{itemize}
		\item How are REE measured in aqueous samples
		\item Will these techniques work for hypersaline, chemically complex brines?
		\item What are the limits of effectiveness for a LLE technique (with respect to brine composition)?
		\item What relationships between REE and bulk solution properties are important for REE fate and transport?
	\end{itemize}
\end{enumerate}


\bibliographystyle{unsrtnat}
\bibliography{Ch1_bib}